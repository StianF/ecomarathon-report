\section{Stian}
Det å jobbe i team har blitt en naturlig del av hverdagen på de fleste arbeidsplasser, 
og jeg føler det derfor er viktig at jeg har bakgrunnskunnskap til å mestre situasjoner som kan oppstå. 
Jeg har som mange andre hørt en del meninger om EiT, og de fleste har dessverre vært negative, 
men ikke annet enn positive om Byggelandsbyen. 
Mye av problemet med andre landsbyer er nok at mange fra Gløshaugen føler a de ikke får utnyttet sin kunnskap i prosjektet, 
noe som kan være veldig demotiverende, og som igjen fører til at motivasjonen for å lære om å jobbe i team blir borte. 
Jeg følte derfor at det var viktig å få  på plass grunnsteinen fra starten, 
derfor var Byggelandsbyen et naturlig valg hvor jeg håpet å jobbe sammen med andre som var motivert for å legge ned en solid innsats, 
og dermed også ende opp med et godt sluttprodukt.   
