Gruppens samspill er veldig viktig for arbeidet som gruppen utfører, men det gjenspeiles også i det ferdige produktet.
Gruppen utviklet seg fra å være enkeltpersoner som kunne hvert sitt område godt, til å være en gruppe personer som har lært å dra nytte av de andres kunnskap.

I starten hadde gruppen få rutiner inne, så for eksempel hvis en beslutning skulle tas, så ble det litt tilfeldig hvordan denne ble tatt. Derfor fant gruppen det veldig naturlig å velge en leder, og ettersom Torgeir pekte seg ut som en som ville ta ansvar, så ble han leder uten diskusjon, nok en gang ble dette veldig tilfeldig, men gruppen var fornøyd med det. I senere saker fant gruppen ut at det å sette opp en vurderingsmatrise var en fin måte å få alles meninger med på en grei og konsis måte, noe som igjen styrket samarbeidet i gruppen og tilhørligheten til produktet som skulle lages.

Ettersom gruppen delte seg i 2 oppgaver, sto vi ovenfor noen utfordringer som andre grupper for det meste slipper. Utfordringen med å inkludere alle i gruppen i begge oppgavene hadde veldig høy prioritet, og ved konstruksjon- og designmessige beslutninger ble hele gruppa med selv om beslutningen ikke var innenfor den enkeltes fagfelt. På denne måten fikk gruppen dratt nytte av den enkeltes synspunkter, ved for eksempel design av brukergrensesnittet var det veldig nyttig at Torgeir og Hans fikk frem sine synspunkter ettersom de har et mer åpent syn på designet enn for eksempel Stian, som ser de tekniske utfordringene fremfor designet.

Det at gruppen delte seg i 2 hjalp sannsynligvis også på konfliknivået i gruppen. Vi hadde ingen konflikter av nevneverdig karakter, dette kan være fordi alle gruppemedlemmene er veldig konkliktskye, men vi tror at dette skyldes at selve utviklingen av hver oppgave ble gjort av personer som til dels tenker veldig likt og ser hverandres positive sider av løsningen, og at vi derfor ikke krangler, men i stedet diskuterer oss frem til en løsning som passer alle.

Fordelingen av oppgaver innad i gruppa falt seg veldig naturlig. På brukergrensesnitt oppgaven, så satte Stian seg ned og lagde et utkast til hvordan systemet skulle være, og ble på den måten selvutnevnt til å lage det, deretetter falt det naturlig at Tore Egil så på telemetrimodulen ettersom det falt innenfor hans fagfelt. Det ble kanskje litt lite oppgaver til Erlend, men dette kan komme av at han hadde veldig mye å gjøre utenom EiT, slik at han ikke hadde tid til å ta på seg altfor mye ansvar, men vi respekterte det, og ettersom vi ikke hadde noen problemer med å gjøre det vi skulle med mindre deltagelse fra Erlend i selve implementasjonen, så gikk det greit. (TODO mer om det, hva følte du da Erlend?). Arbeidet på hengeren delte Torgeir og Hans mellom seg ganske greit. På grunn av Torgeir kompetanse med dreing ble litt mye jobb lagt på Torgeir, men dette jevnet seg ut etterhvert.

Som nevnt ble Torgeir valgt leder for gruppen, men det betyr ikke at han var den eneste sterke personligheten på gruppen. I begynnelsen var alle velding ydmyke, men dette er noe som er veldig typisk for en gruppe i startfasen, og etterhvert som ukene gikk, fant alle sin plass i gruppen. Det viste seg da at alle medlemmene i gruppen er veldig like, alle er veldig åpne for innspill og veldig sammarbeidsvillige, dette er også den største grunnen til at det ikke har oppstått noen konflikter mellom medlemmene av gruppen.

De fleste i gruppen var virket veldig motivert når vi bestemte hvilke oppgaver vi skulle satse på, og vi kjente at dette kom til å bli spennende og utfordrende. Men etter at det hadde gått noen uker og vi hadde kommet godt på vei falt dessverre motivasjonen.
De som lagde brukergrensesnittet fant ut at oppgaven ble enda større enn første forventet. Dette var fordi de fant ut at det som vi trodde bilen ga oss av data aldri ble sendt, så dermed måtte vi plutselig legge ned enda mer tid som ikke var planlagt. 
De som arbeidet med hengeren hadde også kommet godt på vei på dette tidspunktet og var så godt som ferdig med designet når vi fikk beskjed fra ecomarathon-teamet om at vi dessverre ikke fikk NOK 20 000 til å bygge om hengeren, men måtte gjøre det med bare 5 000 NOK. Dette var et stort steg bakover, og vi måtte forkaste mye av det arbeidet vi hadde lagt ned i løpet av de siste ukene. 
Heldigvis lot ikke gruppa seg knekke av dette, selv om vi visste at dette medførte at vi måtte bruke enda mer tid på prosjektet.

Når det gjelder hvordan gruppen fungerte sosialt, så hadde alle medlemmene et godt forhold til hverandre, men vi møttes aldri utenom arbeidsdagene, og dette kan jo både være positivt og negativt. 
Selvfølgelig gir det mindre samhold, men på den annen side står vi mye friere enn hvis vi skulle være en kameratgjeng utenom arbeidstiden. 
Et gruppesamarbeid kan være komplekst nok, og hvis vi da skulle dra sosiale forhold inn i gruppen, så kan dette også være skadelig for gruppen. 
Det kan være at noen av medlemmene kan bli uvenner på fritiden, eller bare bli lei av å se hverandre, eller noen kan bli fryst ut sosialt. 
På den annen side kan noen bli for gode venner og dermed fryse ut andre eller rett og slett ha et altfor kameratslig forhold i arbeidstiden, som igjen kan føre til uproduktivitet.
