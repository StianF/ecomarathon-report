Gruppens medlemmer kjente i utgangspunktet ikke hverandre, så ved det første møtte så var stemningen usikker, men alle på gruppen var veldig innstilt på å bli bedre kjent både faglig og sosialt.
Dette ble delvis løst ved at gruppen ble satt til å skrive ned egenskaper om seg selv i kategoriene praktiske, teoretiske og personlige, selvfølgelig er ikke dette nok for å bli godt kjent, men det var godt på vei.
Etter at dette var gjort fikk gruppen møte supergruppen, de presenterte prosjektet og ga gruppen noen idéer til oppgaver. Ettersom Erlend ikke var tilstede denne onsdagen så ble det vanskelig å bestemme en oppgave, men tanken var at gruppen skulle lage en testbenk for motoren, men Stian var litt skeptisk til om alle kunne jobbe med noe innenfor denne oppgaven. Etter å ha vurdert de forskjellige oppgavene opp mot hverandre ble vi enige om den endelige oppgaven som ble å lage et system som kommuniserte med telemetrimodule i bilen og å forbedre lastefunksjonen i hengeren. Gruppen var enig om at dette var et godt kompromiss ettersom alle fikk jobbe med det de ville, samtidig som de kunne dra nytten av resten av gruppen ved brainstorming og andre praktiske oppgaver.

Tidlig i fasen av prosjektet så gruppen at Torgeir var en sterk personlighet, dermed fant gruppen det naturlig at det kunne være nyttig å velge han til det.

Gruppen satt opp en samarbeidsavtale hvor ting som hva medlemmene forventet av hverandre og hva som skulle skje hvis ikke dette ble fulgt ble skrevet opp. Denne ble ganske godt fulgt i ukene som fulgte.

Mye tid ble brukt til å vurdere forskjellige idéer til hvordan oppgavene skulle løses, og mye tid ble brukt til å finne en brainstormingteknikk som fungerte bra for gruppen. Gruppen består av kun sivilingeniørstudenter, så tankegangen er veldig lik og gruppen liker å finne raske løsninger. Dette ble gruppens svakhet, men også et godt forbedringspotensiale. Gruppen satte derfor opp en plan på hvordan den skulle ta fatt på en oppgave skulle foregå, men denne planen ble etterhvert forkastet fordi den ikke ble fulgt og dermed ikke fungerte i praksis. Senere fant gruppen en god metode fra TODO ref teori som passet gruppen veldig bra. 

Gruppen fant i samarbeid med supergruppen milestones som den skulle følge for å bli ferdig etter planen. Disse gjorde at gruppen fikk god struktur på arbeidet...

I en øvelse ble gruppen satt til å revidere samarbeidsavtalen, dette gjorde at den ble oppmerksom på at det blant annet var viktig å ta opp ting som kanskje ikke det var naturlig å ta opp. Neste arbeidsdag så den nytten av det når Torgeir tok opp at gruppen og da særlig Stian fort blir distrahert til å prate om noe utenfor temaet som diskuteres så fort Torgeir snur ryggen til. Dette ble tatt tak i og gruppen kom til enighet om at hvis flere i gruppen kjenner at det sklir ut, så tar den en liten pause slik at den kan jobbe effektivt når den skal jobbe.

Etterhvert som gruppen kom i gang med oppgavene, så ble den noe splittet, noe som er litt dumt for fellesskapet. Derfor fant gruppen ut at det var lurt å møtes hver onsdag morgen for et statusmøte og for å finne ut hva vi skulle gjøre i løpet av dagen, dette fungerte veldig effektivt og gjorde at fellesskapet kom tilbake. Under dette morgenmøtet får alle i gruppen vite status på arbeidet og dermed hjelpe til der det trengs i løpet av dagen.


