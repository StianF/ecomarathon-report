%Evaluering kunne inneholdt.
%Hva har vi lært.
%Hva har vi lært mest av.
%Hva var bra, hva kunne vært bedre.


På slutten av EiT-prosjektet kan vi se at gruppa har hatt en positiv utvikling fra start til slutt.
Fra å være enkeltpersoner som kan utnytte sitt eget fagfelt godt,
har gruppa utviklet seg til en gruppe personer som kan utnytte hverandres kunnskaper.

Underveis har gruppa observert noen problemer, både med oppgaven og samarbeidet,
og tatt hånd av disse.
Et av de viktigste var å holde gruppen samlet på tross av 2 uavhengige oppgaver, 
og de tiltakne som ble gjennført mener vi har gjort at vi har behold en samlet gruppeidentitet.

Gruppa har også fokusert mye på den kreative prosessen, som var kanskje den største svakheten til gruppa i starten.
Etter å ha tatt ibruk stepladder-metoden merket gruppa en markant forbedring, 
og langt flere og mer hetrogene idèer dukket opp. 
Ved den siste felles landsbydagen ga dette utslag i delt seier i eggkonkurransen.

Vi har tatt lærdom av EiT, og skulle vi ha gjort noe annerledes må det være å oppdage problemer tidligere.
