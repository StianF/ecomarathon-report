%Evaluering kunne inneholdt.

%Hva har vi lært.
%Hva har vi lært mest av.
%Hva var bra, hva kunne vært bedre.


På slutten av EiT-prosjektet kan vi se at gruppa har hatt en positiv utvikling fra start til slutt.
Fra å være enkeltpersoner som kan utnytte sitt eget fagfelt godt,
har gruppa utviklet seg til en gruppe personer som kan utnytte hverandres kunnskaper.

Underveis har gruppa observert noen problemer, både med oppgaven og samarbeidet,
og tatt hånd av disse.
Et av de viktigste var å holde gruppen samlet på tross av 2 uavhengige oppgaver, 
og de tiltakne som ble gjennført mener vi har gjort at vi har behold en samlet gruppeidentitet.

Gruppa har også fokusert mye på den kreative prosessen, som var kanskje den største svakheten til gruppa i starten.
Etter å ha tatt ibruk stepladder-metoden merket gruppa en markant forbedring, 
og langt flere og mer hetrogene idèer dukket opp. 
Ved den siste felles landsbydagen ga dette utslag i delt seier i eggkonkurransen.

Vi har tatt lærdom av EiT, og skulle vi ha gjort noe annerledes må det være å oppdage problemer tidligere.


Gruppen begynte samarbeidet i faget relativt uvitende om hvordan en på en effektiv måte arbeider i team.
Dette kom til utrykk gjennom en treg og lite konstruktiv start på arbeidet de første gruppemøtene.
Ett av problemene gruppen har møtt er prosessen rundt brainstorming.
Gruppen hadde i oppstarten store vansker med å gjennomføre dette uten å bli ensporede og miste fokus.
Det ble også tydelig at gruppen ikke klarte å komme frem til et tilstrekkelig antall ideer eller konsepter gjennom denne prosessen.
Gruppen adresserte dette problemet med å finne litteratur om emnet og forsøkte å bruke en av de kjente modellene for brainstorming, Stepladder teknikken.
Denne teknikken viste seg å fungere veldig godt i gruppen.
En opplevde at antall ideer økte og at alle i gruppen hadde en positiv opplevelse av denne måten å gjennomføre prosessen på.
Gruppen har altså oppdaget et problem, diskutert dette problemet for så å gjennomføre tiltak for å rette dette opp.
Videre ble også effekten av tiltaket vurdert.
Dette ble i hovedsak gjort ved å sammenligne antall gode konsepter som dukket opp under brainstormingen, men og på basis av hver enkelt persons opplevelser av prosessen.

Det ble i gruppen, rimelig tidlig i prosessen, konsensus om at en gruppeleder var nødvendig.
Dette først og fremst på grunn av lite struktur og målrettet arbeid de første to tre ukene.
Gruppen valgte Torgeir som gruppeleder etter at han meldte seg frivillig, dette ble enstemmig vedtatt under ett av møtene.
Det vi ser i ettertid er at gruppen burde ha etablert en definisjon på lederrollen, altså hvilke oppgaver og ansvar gruppelederen skulle ha, før en valgte leder.
Dette ble imidlertid ikke et problem, da alle i gruppen hadde noenlunde lik oppfatning om lederrolles ansvar og forpliktelser.
Gruppeleder tok ansvar der han følte han måtte og delegerte ansvar ut over gruppemedlemmene.
Effekten av at gruppeleder rollen ikke ble tydelig nokk definert kom til syne da gruppen relativt sent i arbeidet fant ut at det var nødvendig at gruppeleder satte opp en innkallelse med dagsplan til hvert møte.
Dette tiltaket kom etter at gruppen hadde opplevd at møtene ble lengre og lengre, samtidig som at svært lite faktisk ble gjort.
Tiltaket viste seg å være svært effektivt på gjennomførelsen av møtene, samlingene ble dramatisk redusert og mye mer ble gjort på mindre tid.
Gruppen har igjen oppdaget et problem, et problem som kunne vert unngått om de riktige tingene hadde blitt gjort tidligere, gjennomført et korrigerende tiltak og evaluert effekten av dette.

Det ble etter vert i prosessen klart for gruppen at en del tverrfaglige utfordringer hadde dukket opp.
Prosjektet vi hadde valgt var i all hovedsak en todelt oppgave, med en mekanisk del og en programmerings del.
Dette kunne fort ha resultert i en meget polarisert gruppe.
Denne problemstillingen ble ganske raskt oppdaget og tatt på alvor.
For å unngå at gruppen ble delt i to ble det også her gjennomført tiltak for å bedre arbeidet i gruppen.

Gruppen har igjennom dette semesteret stått foran en rekke utfordringer og problemer i gruppesamarbeidet.
Disse problemene har blitt poengtert og arbeidet kontinuerlig med å løse.
Vi innser at de fleste problemene mest sannsynlig ikke hadde blitt tatt opp i et annet fag, og vi hadde derfor ikke sett fordelene med å løse disse.
Problemene gruppen har stått foran har kontinuerlig blitt arbeidet med og forsøkt utbedret.
Det å bli oppmerksom på disse problemene har gjort at gruppearbeidet har blitt enklere og ikke minst blitt mer effektivt.
Det å håndtere problem på en profesjonell måte etter hver som de kommer opp i dagens lys er viktig i alle slike prosesser.
Det dette faget har lært oss er at problem skal og må håndteres, videre er det også nødvendig å evaluere de tiltakene som har blitt gjort for å utbedre eller fjerne problemet.
Planlegging av gruppens daglige virke er også noe som er viktig, måten gruppen ledes på er vesentlig for samarbeidet i gruppen.
En hver gruppe møter slike problem, enten om de manifesterer seg i lav effektivitet eller personlige konflikter innad i gruppen.
I dagnes virkelighet er gruppearbeid og team en naturlig del av hverdagen til nesten alle.
Gjennom dette faget har vi lært hvordan vi skal planlegge en gruppeprosess, men og hvordan en håndterer problem i en gruppe.
Dette har vert en   
