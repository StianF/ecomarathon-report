Gruppen har igjennom dette semesteret stått foran en rekke utfordringer og problemer i gruppesamarbeidet.
Når et problem har blitt oppdaget, har det blitt diskutert, for så å gjennomføre tiltak for rette det opp.
Effekten av en tiltakene har også blitt evaluert.

Gruppen begynte samarbeidet i faget relativt uvitende om hvordan en på en effektiv måte arbeider i team.
Dette kom til utrykk gjennom en treg og lite konstruktiv start på arbeidet de første gruppemøtene.
Det var allerede første dag konsensus om at en gruppeleder var nødvendig.
Gruppen valgte Torgeir som gruppeleder etter at han meldte seg frivillig, dette ble enstemmig vedtatt under ett av møtene.
Det vi ser i ettertid er at gruppen burde ha etablert en definisjon på lederrollen, 
altså hvilke oppgaver og ansvar gruppelederen skulle ha, før en leder ble valgt.
Effekten av at gruppelederrollen ikke ble tydelig nok definert kom til syne da gruppen relativt sent i arbeidet fant ut at det var nødvendig at 
gruppeleder satte opp en innkallelse med dagsplan til hvert møte.
Dette tiltaket kom etter at gruppen hadde opplevd at møtene ble lengre og lengre, samtidig som at svært lite faktisk ble gjort.
Tiltaket viste seg å være svært effektivt, og lengden på møtene ble dramatisk redusert.

Gruppa har jobbet mye med den kreative prosessen, som var kanskje den største svakheten til gruppa i starten.
Ved gjennomføring av brainstormer ble idèene som dukket opp for få og homogene.
Som et tiltak forsøkte gruppa forsøkte å bruke en av de kjente modellene for brainstorming: Stepladder-metoden.
En opplevde at antall ideer økte og at alle i gruppen hadde en positiv opplevelse av denne måten å gjennomføre prosessen på.

Under arbeidet dukket noen tverrfaglige utfordringer opp.
Prosjektet vi hadde valgt var i all hovedsak en todelt oppgave, med en mekanisk del og en programvaredel.
Dette kunne fort ha resultert i en meget polarisert gruppe.
Denne problemstillingen ble ganske raskt oppdaget og tatt på alvor.
For å unngå at gruppen ble delt i to ble det også her gjennomført tiltak for å bedre arbeidet i gruppen.
På slutten føler gruppen at de har oppnådd en felles gruppeidentitet.

Vi ser at flere av de problemene vi har møtt på i dette semesteret,
er problemer som vi mest sannsynlig ikke ville ha tatt tak i et annet emne.
Ved å ta tak i disse problemene, har vi opplevd fordelene med å løse disse,
og oppnådd verdifull erfaring om gruppesamarbeid.

På slutten av EiT-prosjektet kan vi se at gruppa har hatt en positiv utvikling fra start til slutt.
Fra å være enkeltpersoner som kan utnytte sitt eget fagfelt godt,
har gruppa utviklet seg til en gruppe personer som kan utnytte hverandres kunnskaper.
