\chapter{Gruppeprosessen}
\section{Ledelse}
En leder kan ha veldig stor betydning for dynamikken i en gruppe, og det finnes
utrolig mye litteratur om nettop lederskap. Et googlesøk på "leadership" gir
over 257 milliarder resultater.
På grunn av den store mengden tilgjengelig litteratur, kan det være vanskelig å velge noe å ta utgangspunkt i.

\subsection{Bakgrunn}
Når EiT startet, så hadde vi ingen leder, og knapt fått hilst på hverandre i
gruppen når vi fikk tildelt en oppgave. Som vi ser av gruppeloggen fra den 19.
januar, så var det ingen til å organisere gruppen, og hele prosessen ble drevet
av initiativ. Det var heldigvis nok initiativ innad i gruppen til å få løst
oppgaven på en tilfredsstillende måte, men vi erkjente umiddelbart behovet for
en leder. Vi bestemte oss derfor i fellesskap for at en leder måtte velges.

\subsection{Lederoppgaver}
Siden den initielle behovet for en leder manifesterte seg som kaos, ble den
første oppgaven å organisere arbeidet til gruppa.
Vi burde muligens ha diskutert bedre på forhånd hvilke oppgaver leder skal ha,
og dokumentert dette i samarbeidsavtalen.

Den 02.02.11 ble Torgeir valgt til gruppeleder. Torgeir meldte seg som frivillig
til å ta lederansvaret, og gruppa ble enige om at dette var greit.

\subsection{Virkninger}
Den umiddelbare virkningen av å ha valgt en leder, var at lederen var klar over
sitt ansvar, og kunne organisere hvordan gruppen jobbet med en oppgave. Vi 
reflekterte ikke så mye på dette underveis, muligens på grunn av det å ha en
leder i en gruppe er noe som er så naturlig at man ikke tenker over
alternativet. Vi ser derimot i ettertid at etter leder ble valgt, ble det langt
mindre problemer med å organisere arbeidet.

Lederen fikk etter hvert tildelt nye ansvarsområder, som f.eks å sette opp en
møteplan for hver onsdag, og lede møtet. Dette og flere tiltak som leder tok
overordnet ansvar for at ble utført har gjort at gruppa jobber langt mer
effektivt enn de første dagene.

Dette med lederskap er ikke noe gruppa som helhet har lagt mye vekt på, og i
samarbeidsavtalen fokuserer vi først og fremst på det ansvaret hver enkelt i
gruppa har.  Gruppa er veldig pragmatisk, og velger det som fungerer. Vi har
brukt mye energi på det med brainstorming, og burde derfor kanskje ha tatt en
brainstorm rundt akkurat dette med lederskap, og deretter tatt en avgjørelse
basert på dette.  Etter en debatt 6. april om akkurat dette, var konklusjonen
at det viktigste var å få en leder som fungerer, slik at man ikke kaster bort
tid som kunne vært brukt på prosjektet. Etter øvelsen på dag 2 var det en klar
oppfattelse av nettopp ineffektivitet og usikkerhet er ting gruppa ikke setter
pris på. Skulle man f.eks ha rulert på lederrollen, ville det ha gått med mye
tid på dette, og man ville risikere uklarhet hvem som til en hver tid har
ansvar. At det ikke har blitt tatt opp problemer underveis gjør at vi
konkluderer med at vi har valgt riktig lederform og leder. 
