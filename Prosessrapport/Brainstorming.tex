\section{Brainstorming}
Brainstorming er noe vi som gruppe har brukt mye tid på. I løpet av prosjektet har vi tatt mange viktige beslutninger ved å gjennomføre en brainstorming. Dette har med andre ord vært noe gruppen har fokusert mye på.
\subsection{Problemer}
Den første brainstormingen vi skulle gjennomføre kom allerede den andre landsbydagen. Da ble det arrangert en aktivitet som gikk ut på at gruppen skulle bygge et sugerørtårn. Gruppeloggen fra denne dagen forklarer godt hvordan dette ble gjennomført: \newline

\emph{“Etter å ha fått utdelt materialene, begynte vi å bygge. Vi tok det første forslaget til byggemetode som ble lagt frem, uten noen diskusjon rundt dette.”} \newline

Problemet var altså ikke hvordan selve brainstormingen ble gjennomført, men det at vi i stor grad ikke gjennomførte en slik prosess, selv om det i forhold til oppgaven var ganske åpenbart at dette var nødvendig. \newline
- Var det noe mer som skjedde under denne øvelsen?
- Hvorfor gikk det så galt?
	- Uvant måte å jobbe på
	- Lite kjent gruppe, lett å unngå konflikter ved å bare ta første og beste løsning
	- Ingen satt struktur på hvordan vi skulle gjennomføre en brainstorming, dermed veldig lett å gli ut


\subsection{Tiltak}
På grunn av denne lite tilfredsstillende gjennomføringen av en brainstorming bestemte vi oss for å konkretisere hvordan vi ville gjennomføre en slik prosess ved senere anledninger. Resultatet av dette var at gruppen lagde en prosessplan (VEDLEGG). Dette var en punktliste med retningslinjer som skulle følges når gruppen utførte oppgaver eller prosjekter. De forskjellige punktene var som følger: \newline
1. Hvert enkeltmedlem i gruppen kommer med minst et eget forslag til løsning. Denne løsningen skal skisseres eller nedskrives på papir, eller annet medium, som datamaskin. Tidsbruken er satt til maksimalt 10 (ti) %. \newline

2. De enkelte gruppemedlemmene presenterer sin egen løsning til de andre i gruppen. Tidsbruken er satt til maksimalt 1 (to) %, for hver person. Total tid for hele gruppen er 5 (fem) % \newline

3. Gruppen skal i felleskap prøve å komme til enighet om en løsning gjennom diskusjon. Gruppen skal prøve å finne en løsning alle har eierskap til, og som alle har et forhold til. Tidsbruken er satt til maksimalt 20 (tjue) %. Om gruppen ikke klarer å finne en løsning innen den satte tidsfristen, skal det gjennomføres en matrisevurdering. Det settes opp en matrise med de forskjellige relevante egenskapene til de forskjellige løsningene. Hver løsing blir vurdert fra 1 (en) - 5 (fem) og hver egenskap blir vektet fra 1 (en) – 5 (fem). Hvert gruppemedlem kan gi et poeng til hver egenskap. Løsning med høyeste sum vil være avgjørende. \newline

4. Gruppen vil etter løsning er valgt, finne den mest konstruktive angrepsvinkelen for å løse problemet på en effektiv måte. Hva som må gjøres først og sist skal bestemmes. Tidsbruk er satt til maksimalt 5 (fem) %. \newline

5. Arbeidsoppgaver skal så fordeles og gjennomføres. Disse skal fordeles på basis av kunnskap, ferdigheter og engasjement. Avhengig av oppgavens natur skal tidsbruken maksimalt være 50 (femti) %, eller så lang tid som gruppen finner nødvendig. \newline

6. Når arbeidsoppgavene er gjennomført skal gruppen i felleskap kvalitetssikre de forskjellige personers arbeid. Her legges det stor vekt på konstruktiv kritikk og saklighet. Avhengig av oppgavens natur skal tidsbruken være maksimalt 10 (ti) %, eller så lenge gruppen finner nødvendig. \newline

Ved å lage denne prosessplanen var tanken at vi nå hadde en konkret måte å gjennomføre brainstorminger på, slik at det ikke skulle gjenta seg at vi gled ut fra, eller droppet hele prosessen. \newline

Første gangen vi skulle bruke denne planen var når vi skulle velge hvilken oppgave vi skulle jobbe med. Heller ikke denne brainstormingen ble gjennomført på en særlig tilfredsstillende måte. Prosessplanen ble fort glemt, og hele prosessen gled ut, noe som kommer klart fram i gruppeloggen: \newline

\emph{“Gruppen begynte nesten umiddelbart å diskutere hvilken av de gitte oppgavene vi skulle velge, uten noen videre form for brainstorming.”} \newline

Etter litt frem og tilbake kom gruppen på prosessplanen, og gjennomførte deler av denne, blant annet at hver enkelt rangerte alle de forskjellige oppgavene. På denne måten fikk alle på gruppen frem sitt syn, og alle de mulige oppgavene ble diskutert, noe som førte til at gruppen endte opp med å endre syn på hva som var den beste oppgaven. Dette viser at vi fikk en mye mer effektiv prosess med en gang vi faktisk fulgte planen vi hadde satt opp. Vårt store problem var dog at vi fort glemte å bruke planen. Dette kan i stor grad skyldes at ingen i gruppen var vant til å jobbe på en slik måte. Det var ingen på gruppen som var vant til å gjennomføre en brainstorming på en strukturert måte, og da ble det veldig lett å gli ut og gjøre det som føltes normalt. \newline

Vi gjennomførte etterhvert en god del brainstorminger hvor tanken var at vi skulle bruke prosessplanen, men ingen av disse ble noen stor suksess. Vi slet hele veien med at vi gled ut fra planen, og selv om vi innførte tiltak som f.eks. å gå gjennom prosessplanen ved starten på hver landsbydag klarte vi aldri å gjennomføre en brainstorming hvor vi fulgte prosessplanen til punkt og prikke. I tillegg til dette slet vi også med at vi ofte kom opp i veldig ensporede diskusjoner. Gruppen hadde en tendens til å fokusere for mye rundt konkrete idéer, og diskutere disse. Dette gjorde at få idéer kom fram, og brainstormingene ble lite vellykket. Dette gjorde at vi fremdeles var langt fra fornøyd med den prosessen vi gjennomførte. \newline

På bakgrunn av alle disse problemene valgte vi å forkaste prosessplanen og finne en ny metode å gjennomføre brainstormingene våre på. Torgeir fant en metode kalt “Stepladder-metoden” (LINK) som vi bestemte oss for å begynne å bruke. Stepladder inneholder fem steg: \newline

1. Presenter oppgaven for hele gruppen. Gi så hver enkelt nok tid til å tenke over problemet og komme opp med sine egne meninger og løsninger.\newline

2. Sett sammen to personer fra gruppen. La de så diskutere problemet sammen.\newline

3. Hent en tredje person fra gruppen, som så presenterer sine løsninger til de to andre. Etter dette skal de andre to legge frem sine løsninger, før alle tre diskuterer de forskjellige idéene. \newline

4. Denne prosessen repeteres ved å hente inn en fjerde person fra gruppen. Det er viktig at det er satt av tid til å diskutere etter at hver nye person har lagt frem sine idéer.\newline

5. Kom fram til en løsning kun etter at alle gruppemedlemmer har fremmet sine idéer.\newline

Den første brainstormingen som ble gjennomført ved hjelp av denne metoden var angående det mekaniske designet på hengeren, som måtte revurderes. I løpet av den innledende idémyldringen kom gruppemedlemmene samlet opp med 14 forskjellige konsepter, noe som kan sies å være en suksess i forhold til den forrige gangen vi gjennomførte en brainstorming på mekanisk design, hvor vi kun kom opp med to konsepter!\newline

Samtidig hadde vi fremdeles en del problemer med prosessen, især når det gjaldt det som gjøres etter den innledende brainstormingen, diskusjon rundt ideene og så beslutningstagning ovenfor hvilken idé som skal gjennomføres. Ettersom det i denne delen av prosessen ikke var like fastsatt hva vi skulle gjøre endte vi fort opp med å diskutere et fåtall av idéene våre igjen.\newline

\subsection{Drøfting}
- Hvorfor klarte vi aldri å bruke prosessplanen skikkelig?
- Var det fordi prosessplanen var for dårlig?
- For lite konkrete første steg, gjorde det lett å dette ut.
- Presentasjon av hverandres idéer var ikke konkret beskrevet nok, gjorde at det ble enklere for    oss å dette ut og diskutere. Det var ingenting som hindret oss i å starte en diskusjon så snart en ide var lagt frem. Når dette da var gjort var det veldig lett å glemme å følge resten av planen, ettersom denne diskusjonen føles intuitiv.
- Motivasjonsfaktor?
- Bruk teori til å forklare hvorfor den tidlige prosessen ikke er idéell i forhold til brainstorming.\newline

- Hva gjør at stepladder fungerte så bra?
- En mer definert idémyldringsprosess, med satte tidsrammer osv. Vi brukte mye mer tid indivduellt her enn før. 
- Fastsatt at alle idéene skulle gås gjennom, forklares for hverandre opp til flere ganger, sørget for at alle fikk mulighet til å fremme sine forslag på en god måte
- Å hente inn en og en person gjorde det vanskelig å bryte ut og starte en lang diskusjon som vi har for vane å gjøre. Dermed ble det holdt fokus på det som skulle gjøres, å presentere alle idéene på en god måte.

