\section{Brainstorming}
Brainstorming er noe vi som gruppe har fokusert mye på. I løpet av prosjektet har vi tatt mange viktige beslutninger ved å gjennomføre en brainstorming. Å gjennomføre en slik prosess på en god måte er ikke alltid like enkelt, noe gruppen etterhvert fikk erfare.
\subsection{Problemer}
Den første brainstormingen vi skulle gjennomføre kom allerede den andre landsbydagen. Da ble det arrangert en aktivitet som gikk ut på at gruppen skulle bygge et sugerørtårn. Gruppeloggen fra denne dagen forklarer godt hvordan dette ble gjennomført:\newline

\emph{“Etter å ha fått utdelt materialene, begynte vi å bygge. Vi tok det første forslaget til byggemetode som ble lagt frem, uten noen diskusjon rundt dette.”} \newline

Problemet var altså ikke hvordan selve brainstormingen ble gjennomført, men det at vi i stor grad ikke gjennomførte en slik prosess, selv om det i forhold til oppgaven var ganske åpenbart at dette var nødvendig. 



\subsection{Tiltak}
På grunn av denne lite tilfredsstillende gjennomføringen av en brainstorming bestemte vi oss for å konkretisere hvordan vi ville gjennomføre en slik prosess ved senere anledninger. Resultatet av dette var at gruppen i fellesskap lagde en prosessplan (Vedlegg \ref{prosessplan}). Det var Torgeir som tok initiativ til å lage denne, og dette var en punktliste med retningslinjer som skulle følges når gruppen utførte oppgaver eller prosjekter. De forskjellige punktene som gjaldt brainstorming var som følger:
\begin{itemize}
\item Hvert enkeltmedlem i gruppen kommer med minst et eget forslag til løsning. Denne løsningen skal skisseres eller nedskrives på papir, eller annet medium, som datamaskin. Tidsbruken er satt til maksimalt 10 (ti) \%.

\item De enkelte gruppemedlemmene presenterer sin egen løsning til de andre i gruppen. Tidsbruken er satt til maksimalt 2 (to) \%, for hver person. Total tid for hele gruppen er 5 (fem) \%
 
\item Gruppen skal i felleskap prøve å komme til enighet om en løsning gjennom diskusjon. Gruppen skal prøve å finne en løsning alle har eierskap til, og som alle har et forhold til. Tidsbruken er satt til maksimalt 20 (tjue) \%. Om gruppen ikke klarer å finne en løsning innen den satte tidsfristen, skal det gjennomføres en matrisevurdering. Det settes opp en matrise med de forskjellige relevante egenskapene til de forskjellige løsningene. Hver løsing blir vurdert fra 1 (en) - 5 (fem) og hver egenskap blir vektet fra 1 (en) – 5 (fem). Hvert gruppemedlem kan gi et poeng til hver egenskap. Løsning med høyeste sum vil være avgjørende.
\end{itemize}
Ved å lage denne prosessplanen var tanken at vi nå hadde en konkret måte å gjennomføre brainstorminger på, slik at det ikke skulle gjenta seg at vi gled ut fra, eller droppet hele prosessen.

Første gangen vi skulle bruke denne planen var når vi skulle velge hvilken oppgave vi skulle jobbe med. Heller ikke denne brainstormingen ble gjennomført på en særlig tilfredsstillende måte. Prosessplanen ble fort glemt, og hele prosessen gled ut, noe som kommer klart fram i gruppeloggen:\newline

\emph{“Gruppen begynte nesten umiddelbart å diskutere hvilken av de gitte oppgavene vi skulle velge, uten noen videre form for brainstorming.”} \newline

Etter litt frem og tilbake kom gruppen på prosessplanen, og gjennomførte deler av denne, blant annet at hver enkelt rangerte alle de forskjellige oppgavene. På denne måten fikk alle på gruppen frem sitt syn, og alle de mulige oppgavene ble diskutert, noe som førte til at gruppen endte opp med å endre syn på hva som var den beste oppgaven. Dette viser at vi fikk en mye mer effektiv prosess med en gang vi faktisk fulgte planen vi hadde satt opp. Vårt store problem var dog at vi fort glemte å bruke planen. Dette kan i stor grad skyldes at ingen i gruppen var vant til å jobbe på en slik måte. Det var ingen på gruppen som var vant til å gjennomføre en brainstorming på en strukturert måte, og da ble det veldig lett å gli ut og gjøre det som føltes normalt.

Vi gjennomførte etterhvert en god del brainstorminger hvor tanken var at vi skulle bruke prosessplanen, men ingen av disse ble noen stor suksess. Vi slet hele veien med at vi gled ut fra planen, og selv om vi innførte tiltak som f.eks. å gå gjennom prosessplanen ved starten på hver landsbydag klarte vi aldri å gjennomføre en brainstorming hvor vi fulgte prosessplanen til punkt og prikke. I tillegg til dette slet vi også med at vi ofte kom opp i veldig ensporede diskusjoner. Gruppen hadde en tendens til å fokusere for mye rundt konkrete idéer, og diskutere disse. Dette gjorde at få idéer kom fram, og brainstormingene ble lite vellykket. Dette gjorde at vi fremdeles var langt fra fornøyd med den prosessen vi gjennomførte.

På bakgrunn av alle disse problemene valgte vi å forkaste prosessplanen og finne en ny metode å gjennomføre brainstormingene våre på. Torgeir fant en metode kalt “Stepladder-metoden” \cite{stepladder} som vi bestemte oss for å begynne å bruke. Stepladder inneholder fem steg: 
\begin{itemize}
\item Presenter oppgaven for hele gruppen. Gi så hver enkelt nok tid til å tenke over problemet og komme opp med sine egne meninger og løsninger.

\item Sett sammen to personer fra gruppen. La de så diskutere problemet sammen.

\item Hent en tredje person fra gruppen, som så presenterer sine løsninger til de to andre. Etter dette skal de andre to legge frem sine løsninger, før alle tre diskuterer de forskjellige idéene.

\item Denne prosessen repeteres ved å hente inn en fjerde person fra gruppen. Det er viktig at det er satt av tid til å diskutere etter at hver nye person har lagt frem sine idéer.

\item Kom fram til en løsning kun etter at alle gruppemedlemmer har fremmet sine idéer.
\end{itemize}
Den første brainstormingen som ble gjennomført ved hjelp av denne metoden var angående det mekaniske designet på hengeren, som måtte revurderes. I løpet av den innledende idémyldringen kom gruppemedlemmene samlet opp med 14 forskjellige konsepter, noe som kan sies å være en suksess i forhold til den forrige gangen vi gjennomførte en brainstorming på mekanisk design, hvor vi kun kom opp med to konsepter!

\subsection{Drøfting}
Det er flere grunner til at prosessplanen vår fungerte så dårlig som den gjorde. Først og fremst var det på grunn av at det rett og slett ikke var en særlig god plan. Spesielt de første stegene av planen var ikke optimale. På punkt 1 hadde vi f.eks. sagt at vi skulle bruke 10 \% av tiden vår på denne individuelle idémyldringen. Dette ble for lite konkret siden vi sjelden hadde bestemt oss for hvor lang en brainstorming skulle være. Siden vi ikke hadde konkrete tidsrammer å forholde oss til endte det fort med at vi brukte lite tid på dette punktet, noe som i ettertid viste seg å være lite smart. Punkt 2, som omhandler presentasjon av idéer, var i tillegg altfor løst definert. Punktet sier bare at alle idéene skal presenteres, ikke hvordan dette skal gjøres. Dermed var det ingenting som hindret oss i å starte en diskusjon så snart en idé var lagt frem. Når dette da var gjort var det veldig lett å glemme å følge resten av planen, ettersom denne diskusjonen føltes intuitiv. Dette er en veldig dårlig måte å gjennomføre en brainstorming på, ettersom bare et fåtall idéer blir diskutert. Det gjør det også vanskeligere for personer i gruppen å komme opp med nye idéer. Dette er beskrevet i en artikkel om brainstorming fra mindtools.com\cite{brainstorming}: \newline

\emph{“Partly this occurs because, in groups, people aren’t always strict in following the rules of brainstorming, and bad group behaviors creep in. Mostly, though, this occurs because people are paying so much attention to other people’s ideas that they're not generating ideas of their own – or they're forgetting these ideas while they wait for their turn to speak. This is called "blocking".”} \newline

Ved å introdusere stepladder-metoden fikk vi i stor grad forbedret alle disse problemene våre. Metodens punkter på hva som skal gjennomføres er mye mer konkret enn hva prosessplanens er. Dette gjorde det plutselig mye enklere for oss å holde fokus på det vi skulle gjøre. Vi satte i tillegg faste tidsrammer vi skulle holde oss innenfor, som f.eks. at vi skulle bruke en halvtime på individuell idémyldring. Dette sørget for at denne delen av prosessen faktisk ble gjennomført på en skikkelig måte. Hver person fikk bedre tid til å sette seg inn i problemet på egenhånd, og også bedre tid til å komme opp med egne idéer. Presentasjonsfasen ble også kraftig forbedret ved å stegvis hente inn nye personer fra gruppen for å presentere sine idéer. Dermed fikk alle på gruppen like god tid til å legge fram sine tanker. Samtidig unngikk vi i stor grad at vi gled ut og begynte å diskutere, som vi hadde for vane å gjøre. Dette skyldtes at det ikke føltes naturlig å starte diskusjoner uten at hele gruppen var samlet, og derfor ble ikke dette gjort før alle gruppemedlemmene hadde fått lagt frem sine idéer. 
