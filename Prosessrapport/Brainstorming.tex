\section{Brainstorming}
Brainstorming er noe vi som gruppe har brukt mye tid på. Brainstorming er en veldig viktig prosess osv. Især for oss som hadde en del viktige beslutninger som skulle tas gjennom brainstorming-prosesser. Brainstorming er en prosess som ofte kan være vanskelig å få til riktig, noe vi fikk erfare ved flere anledninger. (Dårlig intro, fiks senere....)

\subsection{Problemer}
Den første smaken på en slik prosess fikk vi allerede andre dagen, da vi skulle velge hvilken oppgave vi skulle jobbe med. Gruppen hadde allerede laget en prosessplan som skulle følges under en slik prosess, men denne ble fort glemt. (Utdyping angående prosessplanen?) Gruppen endte da fort opp med å bare diskutere mulige løsninger, uten å gjennomføre en skikkelig brainstorming, som man kan se på gruppeloggen: \newline

\emph{“Gruppen begynte nesten umiddelbart å diskutere hvilken av de gitte oppgavene vi skulle velge, uten noen videre form for brainstorming.”} \newline

Etter litt frem og tilbake kom gruppen på prosessplanen vi hadde laget, som skulle brukes i slike situasjoner, og gjennomførte deler av denne, blant annet at hver enkelt rangerte alle de forskjellige oppgavene. På denne måten fikk alle på gruppen frem sitt syn, og alle de mulige oppgavene ble diskutert, noe som førte til at gruppen endte opp med å endre syn på hva som var den beste oppgaven.(Fiks) Dette viser at vi fikk en mye mer effektiv prosess med en gang vi fulgte planen vi hadde satt opp. Vårt store problem var dog at vi fort glemte å bruke denne planen, og dermed gled ut fra brainstorming-prosessen. Dette  kan i stor grad skyldes at ingen i gruppen var vant til å jobbe på en slik måte. Det var ingen på gruppen som var vant til å gjennomføre en brainstorming på en strukturert måte, og da ble det veldig lett å gli ut og gjøre det som føltes normalt. For å bedre nettopp dette ble det bestemt at gruppen skulle gå gjennom prosessplanen om morgenen for å friske opp minnet, og dermed ikke glemme å bruke den igjen. (Teori: Hvorfor folk bare glir ut, blabla) \newline


Gangen etter gjennomførte vi flere brainstorminger. En av disse gikk ut på å finne forskjellige konsepter. Som forrige gang endte det fort med at gruppen skled ut fra prosessen og begynte å på en etterhvert veldig ensporet diskusjon. Gruppen har en tendens til å fokusere veldig på en ide, og diskutere den med en gang den kommer opp. Dermed kommer det veldig få ideer fram, og brainstormingen blir lite vellykket. Vi hadde fremdeles store problemer med å holde oss strukturert i denne prosessen, noe som kommer klart fram av gruppeloggen: \newline

\emph{”I tillegg til dette ble hele aktiviteten preget av en noe ustrukturert diskusjon. Forslag og ideer ble slengt ut, uten at noen nødvendigvis fulgte disse opp med spørsmål eller kommentarer. Dette vil også føre til at en brainstorming blir veldig ensporet.”} \newline

Selvom vi hadde vært veldig klar på at vi måtte sørge for å overholde prosessplanen ved neste anledning klarer vi altså ikke dette. HVORFOR:
 	- Vi hadde jo en plan som skulle følges, hvorfor klarte vi ikke dette?
- Vi hadde vært bevisst gangen før på at vi burde bli mer strukturert, hva gjorde at vi ikke  ble bedre?
- Var prosessplanen for dårlig? Var den konkret nok?
- Fremdeles lite vant til jobbemåten, er det hele grunnen til at dette ikke funket? \newline

De neste gangene ble vi flinkere til å overholde prosessplanen, noe som vises i gruppeloggen: \newline

\emph{“Første punkt på planen var en brainstorming og her ble prosessplanen fulgt. Dette 
innebar blant annet et par minutter med skissering hver for seg før de gikk igang med 
diskusjon og valg av videre løsning. Det at prosessplanen ble fulgt er tegn på at tiltaket vårt 
med å lese gjennom denne på starten av hver dag faktisk virker!”} \newline

Vi slet dog enda med en del problemer, blant annet at vi fremdeles ble veldig ensporet i diskusjonene våre rundt de forskjellige ideene. Dette gjorde at vi fremdeles langt fra var fornøyd med den prosessen vi gjennomførte. \newline

Som en konsekvens av dette ble det bestemt at alle på gruppen skulle finne litteratur på hvordan en god brainstorming skulle gjennomføres. Torgeir fant en metode kalt “Stepladder-metoden” som vi bestemte oss for å begynne å bruke istedetfor den litt mindre suksessfulle prosessplanen. \newline

FORKLAR STEPLADDER \newline

Den første brainstormingen som ble gjennomført ved hjelp av denne metoden var angående det mekaniske designet på hengeren, som måtte revurderes. I løpet av den innledende idémyldringen kom gruppemedlemmene samlet opp med 14 forskjellige konsepter, noe som kan sies å være en suksess i forhold til den forrige gangen vi gjennomførte en brainstorming på mekanisk design, hvor vi kun kom opp med to konsepter! \newline

Grunner til at det funket så bra:
	- En mer definert idémyldringsprosess, med satte tidsrammer osv.
	- Fastsatt at alle idéene skulle gås gjennom, forklares for hverandre opp til flere ganger, sørget for at alle fikk mulighet til å fremme 	sine forslag på en god måte
	- Å hente inn en og en person gjorde det vanskelig å bryte ut og starte en lang diskusjon som vi har for vane og gjøre. Dermed ble det holdt fokus på det skulle gjøres, å  presentere alle idéene på en god måte. \newline

Samtidig hadde vi fremdeles en del problemer med prosessen, især når det gjaldt det som gjøres etter den innledende brainstormingen, diskusjon rundt ideene og så beslutningstaking ovenfor hvilken ide som skal gjennomføres. Ettersom det i denne delen av prosessen ikke var like fastsatt hva vi skulle gjøre endte vi fort opp med å diskutere et fåtall av idéene våre igjen. Det ble bestemt at vi til neste gang skulle lage en strukturert plan for hvordan vi skal vurdere de ulike konseptene vi kommer opp med. \newline

Gjennomførte så en brainstorming på software-biten gangen etter. Fungerte bedre ettersom vi nå hadde litt erfaring ved bruk av denne metoden. \newline

- Hva har vi lært av denne prosessen?
- Hva gjorde at vi ble så enormt mer effektiv ved bruk av den nye metoden?
- Hva burde vi gjort annerledes?
- Hvorfor fungerte ikke prosessplanen? \newline
- 

Pensum:
	- Flett inn hvordan folk normalt oppfører seg i en slik situasjon, som kan vise hvorfor vi  ikke fikk det til de første gangene.
	- Bruk teori til å forklare hvorfor den tidlige prosessen ikke er idéell i forhold til  brainstorming.
	- Forklar Stepladder, bruk teori til å forklare hvorfor denne fungerer såpass bra for oss.

\subsection{Tiltak}

\subsection{Drøfting}

