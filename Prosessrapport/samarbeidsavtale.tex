\section{Samarbeidsavtalen}

TRENGER REFERANSE TIL GROUNDRULES FOR GROUPS!!!

\subsection{Første avtale}
Dag 2 med EiT ble vi bedt om å utforme en samarbeidsavtale som del av en 
gruppeøvelse. Allerede på dag 1 oppdaget vi behovet for å strukturere arbeidet
vårt bedre, og hadde laget et plan for nettopp dette. Det å følge denne planen
ble inkludert i avtalen. Denne avtalen ble basert på en diskusjon rundt hvilke
punkter som måtte være med. Det var her viktig å komme til et konsensus, siden
avtalen ville være bindende for alle medlemmene i gruppa.

Avtalen var først og fremst en dokumentasjon på hva som forventes av hvert
medlem av gruppa, og hvilke sanksjoner man kan forvente hvis man ikke
overholder disse. Avtalen var delt inn i 3 punkter: 
\begin{enumerate}
	\item Hvordan man skulle oppføre seg ovenfor hverandre.
	\item Hvordan kvaliteten på gruppesamarbeidet skal sikres.
	\item Sanksjoner.
\end{enumerate}

\subsection{Revidert avtale}
Den første samarbeidsavtalen var basert på ting vi følte var viktig tidlig i semesteret, og ingen form for teori.
Vi ble 2. mars bedt om å revidere denne avtalen basert på kapittelet "Groundrules for effective groups" i "The Skilled Facilitator" av "Roger Schwartz".
Etter diskusjon i gruppa, ble steg for steg planen for å angripe problemer fjernet til fordel for Stepladder-metoden, som vi hadde hatt gode erfaringer med.
