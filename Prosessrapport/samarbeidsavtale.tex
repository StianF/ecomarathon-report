\section{Samarbeidsavtalen}
Samarbeidsavtalen er et dokument som inneholder regler for hvordan gruppa skal jobbe sammen.
Alle medlemmene av gruppa har forpliktet seg til å følge denne.

\subsection{Første avtale}
Gruppeavtalen ble utformet på dag 2 som en del av en gruppeøvelse.
Allerede på dag 1 oppdaget vi behovet for å strukturere arbeidet vårt bedre, og hadde laget et plan for nettopp dette.
Det å følge denne planen ble inkludert i avtalen. 
I tillegg til arbeidsplanen, hadde vi fra øvelsen flere punkter om hvordan vi ønsket at gruppa skulle fungere.
Det ble debattert hvilke regler man kunne definere for å sikre at disse ønskene ble oppfyllt.
Det var her viktig å komme til et konsensus, siden avtalen ville være bindende for alle medlemmene i gruppa.

Avtalen var først og fremst en dokumentasjon på hva som forventes av hvert
medlem av gruppa, og hvilke sanksjoner man kan forvente hvis man ikke
overholder disse. Avtalen var delt inn i 3 punkter: 
\begin{enumerate}
	\item Hvordan man skulle oppføre seg ovenfor hverandre.
	\item Hvordan kvaliteten på gruppesamarbeidet skal sikres.
	\item Sanksjoner.
\end{enumerate}

\subsection{Revidering}
Gruppeavtalen ble endret noen ganger underveis i prosjektet.
Disse endringene ble gjort for å forbedre samarbeidet i gruppen og øke effektiviteten.

Den første endringen ble gjort 16. februar, etter at gruppa ble delt.
Endringene ble gjort å sikre samarbeidet og gruppetilhørigheten.

Vi ble 2. mars bedt om å revidere samarbeidsavtalen basert på kapittelet "Groundrules for effective groups" i \cite{groundrules}.
Et poeng med dette var at samarbeidsavtalen skulle ha et teoretisk grunnlag, noe det ikke hadde fra før.
Etter diskusjon i gruppa, ble steg for steg planen for å angripe problemer fjernet til fordel for Stepladder-metoden, som vi hadde hatt gode erfaringer med.

Den 16. mars ble det klart at gruppa hadde problemer med å holde fokus. Fra gruppeloggen:

\begin{quotation}
\emph{
	Torgeir foreslo at han som gruppeleder skal lage en møteplan for “møtene” vi har på onsdager. 
	På denne måten vil vi ha foran oss hva vi skal gjøre til enhver tid, 
	slik at det forhåpentligvis blir enklere å holde fokus, 
	og vi unngår at folk plutselig begynner å jobbe med eller diskutere noe annet, 
	som gruppen har hatt en tendens til å gjøre.
}
\end{quotation}
Ved neste møte viste dette seg å fungere, og og følgende ble føyd til i samarbeidsavtalen:
\begin {quotation}
\emph{
	\$ TODO
}
\end{quotation}
