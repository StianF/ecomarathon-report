\section{Samarbeidsavtalen}
Samarbeidsavtalen er et dokument som inneholder regler for hvordan gruppa skal jobbe sammen.
Alle medlemmene av gruppa har forpliktet seg til å følge denne.

\subsection{Første avtale}
Allerede på dag 1 oppdaget vi behovet for å strukturere arbeidet vårt bedre, og hadde laget et plan for nettopp dette.
Da gruppeavtalen skulle utformes på dag 2, ble derfor det å følge denne inkludert i avtalen.

Fra gruppeøvelsen hadde gruppemedlemmene fullført de følgende setningene.
\begin{itemize}
\item Mine beste erfaringer med teamarbeid
\item Det aller viktigste for at jeg skal trives
\item Det aller viktigste for at jeg skal yte best
\item Det som iriterer meg grenseløst
\end{itemize}

Ut ifra hva medlemmene hadde svart på dette ble det debattert hvordan man kunne sikre ytelse og trivsel, og hindre irritasjon.
Det ble foreslått regler for å sikre dette som skulle være del av en samarbeidsavtale for gruppa.
Det var her viktig å komme til et konsensus, siden avtalen ville være bindende for alle medlemmene.

Avtalen var først og fremst en dokumentasjon på hva som forventes av hvert
medlem av gruppa, og hvilke sanksjoner man kan forvente hvis man ikke
overholder disse. Avtalen var delt inn i 3 punkter: 
\begin{enumerate}
	\item Hvordan man skulle oppføre seg ovenfor hverandre.
	\item Hvordan kvaliteten på gruppesamarbeidet skal sikres.
	\item Sanksjoner.
\end{enumerate}

\subsection{Revidering}
Gruppeavtalen ble endret noen ganger underveis i prosjektet.
Disse endringene ble gjort for å forbedre samarbeidet i gruppen og øke effektiviteten.
Den første endringen ble gjort 16. februar, etter at gruppa ble delt.
Følgende ble da lagt til i avtalen:
\begin{quotation}
Tiltak for å styrke gruppens tverrfaglige samarbeid:
\begin{enumerate}
\item Gruppen skal, så langt det lar seg gjøre, alltid sitte sammen under gruppemøtene på onsdager.
\item På starten og slutten av hvert gruppemøte skal alle deltagerne avlegge en statusrapport til resten av gruppen.
\item Hver delgruppe har ansvar om å til enhver tid holde den andre delgruppen orientert om hvordan det går med deres del av prosjektet. 
\item Gruppemedlemmene skal engasjere seg mer på tvers av fagretningene og inndelingen av undergrupper og komme med innspill til hverandre. 
\end{enumerate}
\end{quotation}

Vi ble 2. mars bedt om å revidere samarbeidsavtalen basert på kapittelet ``Groundrules for effective groups'' i \cite{groundrules}.
Et poeng med dette var at samarbeidsavtalen skulle ha et teoretisk grunnlag, noe det ikke hadde fra før.
Etter diskusjon i gruppa, ble Prosessplanen fjernet til fordel for Stepladder-metoden, som vi hadde hatt gode erfaringer med.

Den 16. mars ble det klart at gruppa hadde problemer med å holde fokus. Fra gruppeloggen denne dagen kan vi lese følgende:

\begin{quotation}
\emph{
	``Torgeir foreslo at han som gruppeleder skal lage en møteplan for “møtene” vi har på onsdager. 
	På denne måten vil vi ha foran oss hva vi skal gjøre til enhver tid, 
	slik at det forhåpentligvis blir enklere å holde fokus, 
	og vi unngår at folk plutselig begynner å jobbe med eller diskutere noe annet, 
	som gruppen har hatt en tendens til å gjøre.''
}
\end{quotation}
Ved neste møte viste dette seg å fungere, og og følgende ble føyd til i samarbeidsavtalen:
\begin {quotation}
\emph{
	For å sikre at gruppen fungerer som tiltenkt skal leder:
	\begin{enumerate}
		\item Ha et overordnet ansvar for at gruppen kommer i gang med de aktiviteter som skal gjennomføres.
		\item Dersom ikke annet er bestemt, sette opp en møteplan før gruppen møtes.
		\item Lede møtet etter møteplanen.
	\end{enumerate}
}
\end{quotation}
