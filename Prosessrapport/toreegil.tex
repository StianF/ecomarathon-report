\section{Tore Egil}
Informasjonsformidlingen om EiT har vært generelt dårlig, men byggelandsbyen
er et unntak. Ladsbylederen var flink til å reklamere for sin landsby, og hadde
god konkret informasjon tilgjengelig. 

Jeg hadde en del forventninger før dette prosjektet. Å få benyttet mine faglige
kvalifikasjoner var viktig, men viktigst det å få jobbe med noe litt håndfast.
Det var derfor en del spenning om hva den faktiske oppgaven ville gå ut på.

Å jobbe i et tverrfaglig team kan by på utfordringer, men også nye
læringsmuligheter. En viktig motivasjon før prosjektet var derfor muligheten til
å utveksle nye kunnskaper med gruppas medlemmer.


