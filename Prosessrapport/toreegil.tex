\section{Tore Egil}
Når jeg skulle velge landsby, hadde informasjonen vært generelt dårlig.
Det var heldigvis noen unntak, og byggelandsbyen var en av disse.
Landsbylederen var flink til å reklamere for sin landsby, og hadde god konkret informasjon tilgjengelig. 

Jeg hadde selvsagt noen forventninger før dette prosjektet. 
Å få benyttet mine faglige kvalifikasjoner var viktig, men viktigst det å få jobbe med noe litt håndfast.
På bakgrunn av informasjonen om byggelandbyen, forventet jeg også å få jobbe med noe litt hardware-nært.
Når det ble klart hvilken gruppe jeg havnet på, så ble det at det ferdige produktet var noe som ecomarathon-teamet kunne ha nytte av et viktig poeng.
Det ble derfor en del spenning om hva den faktiske oppgaven ville gå ut på.

Å jobbe i et tverrfaglig team kan by på utfordringer, men også nye læringsmuligheter. 
En viktig motivasjon før prosjektet var derfor muligheten til å utveksle nye kunnskaper med gruppas medlemmer.


