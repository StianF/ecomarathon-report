\section*{\centering Oppsummering}
Denne rapporten har som hensikt å gi leseren et innblikk i gruppens samarbeid gjennom prosjektperioden. Prosjektet har til tider vært krevende, og satt store krav til både gruppen som helhet samt enkeltpersonene innad i gruppen. Til tross for dette har gruppen fungert overraskende bra. Vi mener selv at vi har hatt et godt samspill oss imellom, noe som også har vært med på å sikre et godt prosjektmessig resultat.\\\\
Gjennom å lese denne rapporten vil leseren blant annet få en oversikt over hvilke problemer vi har støtt på underveis. Et av disse er utfordringene rundt det å jobbe med en todelt oppgave, noe som naturligvis også har ført til en todeling av gruppen. Dette har fått innvirkninger på gruppens samhørighet og tilhørighet til prosjektet, og dermed vært med på å prege prosjektperioden. Siden dette har vært et aktuelt tema gjennom hele prosjektperioden har vi valgt å bruke mye energi på nettopp dette. Rapporten tar for seg hvilke utfordringer som oppstår rundt et slikt problem, samt effekten av de gjennomførte tiltakene for å forsøke å bedre på situasjonen.\\\\
Det viste seg tidlig i prosjektarbeidet at gruppen dessverre hadde store problemer med å gjennomføre en god brainstorming. Vi så gjentatte ganger at vi hadde veldig lett for å låse oss fast på en ide, og utelukke resten. På denne måten ble mange gode ideer forkastet allerede før de ble vurdert. Dette var en stor svakhet gruppen hadde, og vi bestemte oss derfor for å begynne å lete etter løsninger på problemet. Det skulle vise seg at Stepladder-metoden ble løsningen på våre problemer.\\\\
Rapporten vil også gi leseren et innblikk i noen utvalgte øvelser som har blitt gjennomført i regi av fasilitator. Noen av disse øvelsene kunne virke ganske meninsgløse når vi gjennomførte de, og vi slet derfor litt med å se nytten av disse. I ettertid har det likevel vist seg at disse øvelsene ikke var så unyttige som først antatt.



