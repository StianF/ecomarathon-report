\section{Øvelser}
Det ble gjennom hele semesteret arrangert diverse øvelser både av landsbyleder og av fasilitatorene. I dette avsnittet vil det bli presentert de av øvelsene som gruppen eller den enkelte følte de hadde mest utbytte av.
\subsection{Bli-kjent-øvelser}
Den første dagen vi møttes ble det arrangert noen enkle bli-kjent-øvelser. Noen arrangert av fasilitatorene og noen av de enkelte gruppene. Et utdrag fra Torgeir sin personlige logg etter denne dagen gir oss et inntrykk av hvordan han opplevde dette.
\begin{quote}
En helt bortkastet dag. Hvorfor i all verden skal NTNU bruke tid, penger og energi på dette? Det minner meg og første klasse på barneskolen, er ikke vi kommet lengre nå? Skjønner at de vil vi skal bli bedre kjent, og at dette kanskje kan gjøre at vi fungerer bedre som en gruppe, men de får da være grenser. 
\end{quote}
I etter tid har gruppen skjønt at disse øvelsene faktisk var svært så nyttige. Gjennom semesteret har vi blant annet lært om gruppeteori. Gruppen har satt seg inn i faser i MRPI modellen\footnote{Her må vi ha en kildehennvisning}, i denne modellen er gruppearbeidet delt inn fire faser: “Bli-kjent-fasen”, “Standpunktfasen”, “Overglattingsfasen” og “Frigjøringsfasen”. Denne modellen sier at bli-kjent-fasen er kjennetegnet av uklare mål eller individuell målsetning, ikke nødvendigvis et felles mål for gruppen. Rolleinndelingen er ofte uklar og prosedyrene er ofte tilfeldige og at de enkelte gruppemedlemmene forholder seg til de andre på en usikker og gjerne overfladisk måte. I ettertid ser vi at disse øvelsene har gjort at denne lite produktive fasen av gruppearbeidet har gått på en raskere måte en om vi ikke hadde gjennomført dem. Torgeir har etter gruppemøte den 16.02 skrevet.
\begin{quote}
Har vel egentlig innsett at gruppen har et behov for å bli mer effektive. Så langt har det ikke blitt gjort så veldig mye, fire uker ut i prosjektet og ingenting vesentlig har blitt gjort. Kan jo tenkes at dette har med innstillingen til de enkelte å gjøre, samt en litt vag oppgavebeskrivelse. Hoved poenget er at vi bør komme i gang snart. 
\end{quote}
Ser vi dette i lys av de gjennomførte bli-kjent-øvelsene er det for oss ganske så klart at gruppen kunne vert mer effektive om den ineffektive bli-kjent-fasen hadde blitt unngått. Det er i utgangspunktet derfor lurt å ha slike øvelser, eller en eller annen form for sosial aktivitet de første dagene etter at en gruppe er startet. 
\subsection{Hatteøvelsen}
En øvelse som ble gjennomført på gruppenivå av fasilitator var en øvelse kalt hatteøvelsen.  Dette var en relativt liten øvelse men som til gjengjeld avslørte noe nytt om gruppen. Under øvelsen skulle den enkelte skrive ned to positive og en negativ tin om gruppen, legge denne i en hatt. Etter dette skulle fasilitator lese opp lappene en og en, gruppen skulle deretter diskutere påstanden. 
En av tingene som kom opp under denne øvelsen kom veldig overaskende på mange i gruppen. Det som kom frem var at noen i gruppen føler at han ikke blir hørt. Gruppen er i utgangspunktet sammensatt av en rekke sterke personligheter, dette kan nokk være en del av grunnen til at det lett kon oppstå slike situasjoner. 
Under diskusjonen som fulgte ble dette diskutert, gruppen innser at noen sterke personligheter kan ha overskygget andre i gruppen. Selv om dette ikke på noen som helst måte er med overlegg. Dette er en situasjon som lett kunne ført til en urolig atmosfære innad i gruppen, og er derfor viktig å diskutere. Vi føler at en slik øvelse kan virke positivt på en gruppes måte å arbeide på, da ting lettere kommer frem til overflaten ved anonymitet. NOEN skriver i den personlige loggen etter denne dagen.
\begin{quote}
Det var veldig overaskende at en person i gruppen av og til føler at han ikke blir hørt, dette har ikke jeg merket. Jeg syntes gruppen virkelig har tatt initiativ til å unngå nettopp dette ved å gjennomføre brainstorminger ved hjelp av stepladder metoden. På en annen side innser jeg at enkelte sterke personligheter i gruppen lett kan akkumulere en slik situasjon. 
\end{quote}