\chapter{Samarbeidsavtale}
%Paragraf 1-------------------------------------------------------------------------------------------------
\section*{§1 - Personlig oppførsel}
Alle gruppemedlemmene skal ovenfor hverandre:
\begin{enumerate}
\item Ha en respektfull, profesjonell og god holdning.
\item Forsikre at alle antagelser og slutninger er riktige, dette gjelder for alle parter i en samtale.
\end{enumerate}
%Paragraf 2-------------------------------------------------------------------------------------------------
\section*{§2 - Struktur og orden}
Alle gruppemedlemmene skal for å sikre kvaliteten på arbeidet i gruppen:
\begin{enumerate}
\item Komme til avtalt tidspunkt, om ikke annet er avtalt er dette tidspunktet 09.15.
\item Varsle minst en person i gruppen så tidlig som mulig om en ikke kan komme til avtalt tidspunkt.
\item Gjennomføre utdelte arbeidsoppgaver etter beste evne.
\item Varsle tidlig om arbeidsoppgaver ikke kan utføres innen gitte tidsfrister. 
\item Søke hjelp av de andre i gruppen eller andre utenforstående dersom arbeidsoppgaver ikke kan utføres innen for gitte tidsfrister.
\item Jobbe effektivt og strukturert innenfor avsatt arbeidstid i gruppen.
\item Forsikre seg om at en er siker på hva en skal gjøre, dersom usikker, rådføre seg med gruppen, eller andre utenforstående.
\item Sørge for at gruppens interesser kommer foran personlige synspunkt og meninger gjennom fornuftig argumentasjon og kritisk tankegang.
\item Sørge for at alle resonnement blir forklart og for at en har forstått disse i sin helhet. Dette er et gjensidig ansvar.
\item Dele all relevant informasjon med resten av gruppen.
\end{enumerate}
%Paragraf 3-------------------------------------------------------------------------------------------------
\section*{§3 - Gruppens arbeidsmåte}
Gruppen skal bruke følgende verktøy i følgende situasjoner:
\begin{enumerate}
\item Stepladdermetoden i brainstormingsituasjoner.
\item Vurderingsmatrise før slutninger tas.
\end{enumerate}
Tiltak for å styrke gruppens tverrfaglige samarbeid:
\begin{enumerate}
\item Gruppen skal, så langt det lar seg gjøre, alltid sitte sammen under gruppemøtene på onsdager.
\item På starten og slutten av hvert gruppemøte skal alle deltagerne avlegge en statusrapport til resten av gruppen.
\item Hver delgruppe har ansvar om å til enhver tid holde den andre delgruppen orientert om hvordan det går med deres del av prosjektet. 
\item Gruppemedlemmene skal engasjere seg mer på tvers av fagretningene og inndelingen av undergrupper og komme med innspill til hverandre. 
\end{enumerate}
%Paragraf 4-------------------------------------------------------------------------------------------------
\section*{§4 - Lederoppgaver}
For å sikre gruppen fungerer som tiltenkt skal leder:
\begin{enumerate}
\item Ha et overordnet ansvar for at gruppen kommer i gang med de aktiviteter som skal gjennomføres.
\item Dersom ikke annet er bestemt, sette opp en møteplan før gruppen møtes.
\item Lede møtet etter møteplanen.
\end{enumerate}
%Paragraf 5-------------------------------------------------------------------------------------------------
\section*{§5 - Sanksjoner}
Dersom en person i gruppen ikke overholder §1-2 skal følgende sanksjoner tas i bruk:
\begin{enumerate}
\item Ved forsinkelse til oppmøte med mer en 15 min (femten minutter). Skal vedkommende legge 20 Nok (tjue norske kroner) i velferdspotten. 
\item Generelle brudd på §1-2 blir ført i gruppeloggen. 
\end{enumerate}
