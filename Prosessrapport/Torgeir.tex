\section{Torgeir}
Jeg visste svært lite sikkert om EiT før semesterstart, og hadde derfor gjort meg opp få eller ingen meninger om dette faget. Det lille jeg hadde fått med meg og visste med sikkerhet var at faget var et tverrfaglig prosjekt og at det var obligatorisk oppmøte hver onsdag. Jeg bestemte meg ganske lenge før semesterstart at jeg ville være i Byggelandsbyen. For meg ble dette valget tatt på grunnlag a taktikk, men mest interesser. Jeg antok at siden man spesifikt måtte sende mail til landsbyleder for å få være med, måtte de som ble med i denne landsbyen nødvendigvis være godt motivert for dette prosjektet. På denne måten kunne jeg sikre meg at gruppearbeidet ikke ble forstyrret av motivasjonsproblemer. Siden jeg i tillegg er en person med sterk praktisk tilnærming til problemløsning, tenkte jeg at Byggelandsbyen måtte være det beste valget vor meg. Av personlige erfaringer vet jeg at jeg har en sterkt resultatorientert personlighet, og sikter alltid på toppen. 

Mine forventninger til gruppearbeidet var at jeg først og fremst kunne få bruk for mine praktiske kunnskaper. Jeg forventet også å få nye erfaringer i det å jobbe sammen i en tverrfaglig gruppe, da jeg tidligere ikke har så mange slike erfaringer fra før.   

Karaktermålet mitt for dette faget er derfor en karakter i toppsjiktet. Alle som går på NTNU har kommet hit for en grunn og er svært motiverte og ikke minst dyktige. På dette grunnlag tok jeg den slutning at gruppen mest sannsynlig ville bestå av personer med lik grad av motivasjon og karakterforventninger.     
