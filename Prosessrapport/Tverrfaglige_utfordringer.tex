\section{Tverrfaglige utfordringer}
Gruppen har vært todelt gjennom hele prosjektperioden. Årsaken til dette er at vi har jobbet med to vidt forskjellige oppgaver. Torgeir og Hans har jobbet med det tekniske, det vil si på tilhengeren til Shell eco marathon. Stian, Tore Egil og Erlend har jobbet med software for å få sanntidsinformasjon sendt fra bilen og inn til pit.

\subsection{Problemer}
Siden vi har jobbet med to såpass ulike oppgaver, har det ofte vært naturlig for gruppen å dele seg. Dette ser vi et godt eksempel på i Gruppelogg 5: \newline

\emph{``Etter møtet delte gruppen seg. Tore Egil, Stian og Erlend jobbet med software-biten. Torgeir
og Hans jobbet med den mekaniske delen av oppgaven.''}\newline

Dette skjedde opptil flere ganger. Ofte hadde det rent praktiske årsaker, som for eksempel at Torgeir og Hans skulle jobbe på verkstedet. Men noen ganger delte vi oss også uten at dette var tilfelle. Etter en kort tid begynte gruppen som helhet å merke at dette hadde en del negative innvirkninger på gruppesamarbeidet og gruppens dynamikk. 
Det første vi merket var at det var vanskelig å holde oversikt over hva den andre halvdelen av gruppen jobbet med. Det var vanskelig for den ene delen av gruppen å vite om de andre hadde støtt på problemer, om de trengte hjelp til noe, hvordan de gjorde det i forhold til tidsplanen og lignende. Det at vi ikke jobbet sammen gjorde det også verre å få til en god kommunikasjon mellom gruppemedlemmene og det ble vanskeligere å gjøre helt enkle ting som for eksempel å avtale et gruppemøte. \newline

Denne måten å jobbe på gjorde det også vanskelig for gruppen å få et ordentlig eierskap til prosjektet som helhet. Et godt eierskap til prosjektet fører med seg en del gode frukter. Ikke bare styrker det gruppens samhold, det er også med å sikre kvaliteten på arbeidet. Vi ble mest fokusert på det vi selv jobbet med, og glemte ofte at det var en annen del av gruppen som jobbet med noe helt annet. 
Gruppeinndelingen påvirket også samholdet mellom gruppemedlemmene. Gruppefølelsen uteble, og det var ofte vanskelig å føle at gruppen jobbet sammen mot et felles mål.  

\subsection{Tiltak}
Professor Surinder Kahai ved Binghamton University i New York uttalte i 2007 følgende:\newline

\emph{``I have found that when team members think of the project as “their project,” they have a positive attitude towards it and they put in a lot more effort. Their self-esteem is higher and so is their commitment to project’s success. ''}  (www.leadingvirtually.com)\newline

Dette utsagnet beskriver på mange måter de problemer vi har beskrevet ovenfor, og ut ifra det vi kunne finne av litteratur virket det som om løsningen på de tverrfaglige problemene våre lå i det la gruppen som helhet få større eierskap til prosjektet. \newline

Dette kommer også frem fra Gruppelogg 6:\newline
 \emph{``Når gruppen jobber med to forskjellige, nesten uavhengige prosjekter, blir det ikkel ike lett å få det fulle og hele eierskap til alt. Dette er viktig både for samhold i gruppen men og, kvaliteten på arbeidet. Tore Egil påpekte under en diskusjon i gruppen angående nettopp dette, at hele gruppen burde samles, slik at de som jobber med den mekaniske delen kan få innspill og kommentarer fra de som jobber med software delen, og vise versa.''}\newline

Resten av gruppen var enige i dette, og siden teorien også støttet opp under dette ble gruppen enig om å gjennomføre en del strakstiltak:

\begin{enumerate}
\item Gruppen skulle, så langt det lot seg gjøre, alltid sitte sammen under gruppemøtene på onsdager.
\item På starten og slutten av hvert gruppemøte skulel alle deltageren avlegge en  statusrapport til resten av gruppen.
\item Hver delgruppe hadde ansvar om å til enhver tid holde den andre delgruppen orientert om hvordan det gikk med deres del av prosjektet.
\item Gruppemedlemme skulle angasjere seg mer på tvers av fagretningene og inndelingen av undergrupper og komme med innspill til hverandre. 
\end{enumerate}

\subsection{Drøfting}
Etter en liten tid merket vi at tiltakene våre fikk gode konsekvenser. Vi jobbet bedre sammen og følte også at det var bedre samhold i gruppen. 
Det at vi bestemte oss for å sitte sammen å jobbe så fremt det lot seg gjøre, medførte at det var enklere for gruppens medlemmer å få et overblikk over hvordan det gikk med arbeidet til resten av gruppen. Det var også lettere å komme med innspill og ideer på tvers av fagretningene. \newline

To uker etter at tiltakene ovenfor ble vedtatt, skrev Hans følgende i sin personlige logg:\newline

\emph{``I dag var det inspirerende å jobbe sammen med resten av gruppen. Jeg merker stor forskjell i min egen og gruppens motivasjon nå i forhold til for bare en uke siden. Gruppen jobber mye bedre sammen og det virker som om alle er interessert i å yte sitt beste''}\newline

I ettertid er det lett å se at disse tiltakene burde vært gjennomført tidligere, men for at dette skulle vært mulig måtte vi også ha identifisert problemene tidligere i gruppeprosessen. Siden vi tidlig ble klar over at vi kom til å jobbe som en todelt gruppe, burde vi ha lest teori på dette med det samme vi gikk igang med prosjektet. På denne måten ville vi vært bedre forberedt på de utfordringer vi kom til å møte og også vært istand til å gjennomføre tiltakene før problemene oppstod. 