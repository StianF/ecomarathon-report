\chapter{Prosessplan}
\label{prosessplan}
{\bf 26.01.11}\\
{\bf NTNU}\\
{\bf 7091}\\
{\bf Trondheim}\\

{\bf Prosessplan. Eksperter i team, Shell Eco Maraton bil}\\
For å kvalitetssikre arbeidet i gruppen, skal gruppen følge punktene 1 til 6 under når oppgaver eller prosjekter gjennomføres. Disse punktene er ment som retningslinjer ikke absolutte restriksjoner eller regler. Tidsbruken er oppført som prosenter av den totale tiden gruppen har på de forskjellige delene, eller den totale tiden av oppgaven eller prosjektet
\begin{itemize}
\item Hvert enkeltmedlem i gruppen kommer med minst et eget forslag til løsning. Denne løsningen skal skisseres eller nedskrives på papir, eller annet medium, som datamaskin. Tidsbruken er satt til maksimalt 10 (ti) %.
\item De enkelte gruppemedlemmene presenterer sin egen løsning til de andre i gruppen. Tidsbruken er satt til maksimalt 2 (to) \%, for hver person. Total tid for hele gruppen er 5 (fem) \%
\item Gruppen skal i felleskap prøve å komme til enighet om en løsning gjennom diskusjon. Gruppen skal prøve å finne en løsning alle har eierskap til, og som alle har et forhold til. Tidsbruken er satt til maksimalt 20 (tjue) \%. Om gruppen ikke klarer å finne en løsning innen den satte tidsfristen, skal det gjennomføres en matrisevurdering. Det settes opp en matrise med de forskjellige relevante egenskapene til de forskjellige løsningene. Hver løsing blir vurdert fra 1 (en) - 5 (fem) og hver egenskap blir vektet fra 1 (en) - 5 (fem). Hvert gruppemedlem kan gi et poeng til hver egenskap. Løsning med høyeste sum vil være avgjørende.
\item Gruppen vil etter løsning er valgt, finne den mest konstruktive angrepsvinkelen for å løse problemet på en effektiv måte. Hva som må gjøres først og sist skal bestemmes. Tidsbruk er satt til maksimalt 5 (fem) %
\item Arbeidsoppgaver skal så fordeles og gjennomføres. Disse skal fordeles på basis av kunnskap, ferdigheter og engasjement. Avhengig av oppgavens natur skal tidsbruken maksimalt være 50 (femti) \%, eller så lang tid som gruppen finner nødvendig.
\item Når arbeidsoppgavene er gjennomført skal gruppen i felleskap kvalitetssikre de forskjellige personers arbeid. Her legges det stor vekt på konstruktiv kritikk og saklighet. Avhengig av oppgavens natur skal tidsbruken være maksimalt 10 (ti) \%, eller så lenge gruppen finner nødvendig.
\end{itemize}

Etter ferdig gjennomførte arbeidsoppgaver og kvalitetssikring kan det bli nødvendig med ytterligere arbeid for å sikre tilstrekkelig kvalitet. Det legges stor vekt på at gruppen i felleskap skal finne de beste løsningene, både med tanke på kvalitet og tidsbruk. Gruppen skal derfor, før det deles ut nye arbeidsoppgaver, være fornøyd med det arbeidet hver enkelt har gjort. Kvalitetssikring og vurdering av arbeid utført av enkeltpersoner skal alltid gjennomføres i ettertid.
