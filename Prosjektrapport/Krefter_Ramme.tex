\section{Beregning av opplagerkrefter}
Det er nødvendig å bestemme kreftene som opptrer i rammens opplager for å kunne verifisere designet


\begin{figure}[H]
\begin{center}
\leavevmode
\includegraphics[width=0.8\textwidth]{images/Bilden_1}
\end{center}
\caption{Krefter ramme}
\label{fig:Krefter}
\end{figure}


Kraften $F_3$ tilsvarer halvparten av bilens vekt, siden den blir fordelt på rammens to sider. I denne rapporten er det beregnet med en vekt på bilen tilsvarende 100 kg. Det er videre antatt at denne kraften virker midt på rammen (bilens tyngdepunkt på midten). Rammens egenvekt er neglisjert. \\



Momentlikevekt om Punkt 1:

\begin{equation}
\sum{M_1}=0
\end{equation}

\begin{equation}
F_3(L_1+L_2)-F_2\cdot L_1=0
\end{equation}
 
\begin{equation}
F_2=\frac{F_3(L_1+L_2)}{L_1}
\end{equation}

\begin{equation}
F_2=\frac{F_3(L_1+L_2)}{L_1}
\end{equation}\\\\

Kraftlikevekt i Y-retning:


\begin{equation}
\sum{F_Y}=0
\end{equation}

\begin{equation}
F_2-F_1-F_3=0
\end{equation}

\begin{equation}
F_2-F_1-F_3=0
\end{equation}
 
\begin{equation}
F_1=F_2-F_3
\end{equation}


\begin{equation}
F_1=\frac{F_3(L_1+L_2)}{L_1}-F_3
\end{equation} \\

Graf som viser forholdet mellom $L_1$ og opplagerkreftene $F_1$ og $F_2$:

\begin{figure}[H]
\begin{center}
\leavevmode
\includegraphics[width=1.0\textwidth]{images/Bilden_2}
\end{center}
\caption{$F_1$ og $F_2$ som en funksjon av $L_1$}
\label{fig:Krefter}
\end{figure}

Setter minste verdi av $L_1$ lik 0.5 m. Vil da få opplagerkrefter på $F_1$ lik 1230 N og $F_2$ lik 1720 N. Dette er de største opplagerkreftene som kan oppstå, og det er disse verdiene som brukes videre i styrkeberegningene.


