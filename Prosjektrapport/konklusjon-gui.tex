Det endelige resultatet som er beskrevet i del \ref{sec:losning} møtte alle de høyest prioriterte kravene så langt det var teknisk mulig.
Det ble aldri implementert kommunikasjonsmuligheter fordi dette viste seg at ville medføre ny hardware som teamet ikke hadde budsjetert, og en eventuell løsning ville vært så komplisert at det sannsynligvis ikke ville blitt ferdig i tide.
Løsningen har blitt grundig testet i samarbeid med Ecomarathon-teamet, og blitt tatt veldig godt i mot av dem.

En fra gruppen vil fortsette å være i kontakt med teamet i tiden fremover mot løpet, slik at teamet vet hvordan systemet skal brukes, og at eventuelle problemer som måtte oppstå lett kan løses.

Det har blitt konstruert, med utgangspunkt i et allerede eksisterende design, en rammekonstruksjon, som på en sikker og enkel måte muliggjør transport av DNV Fuel Fighter, NTNU’s bidrag til Shell Eco Maraton løpet. Budsjett rammen på 5000,- har blitt overholdt med god margin. Videre har også kravene fra oppdragsgiver, etter vårt syn, blitt overholdt. Den nye rammekonstruksjonen er designet for at det skal bli enklere å få bilen inn i henger, uten å skade bil eller personell på noen som helst måte. 
Alle de nye delene i designet har blitt verifisert med FE-analyser og godkjent med akseptable sikkerhetsfaktorer mot flyt i materialet. 
Siden det nye designet i stor grad baserer seg på det gamle, har det blitt vanskelig å anslå med rimelig sikkerhet hvor mye rammen kan lastes med. Dette på grunn av usikkerheten en konstruksjon i tre medfører. Tenker her spesielt på problemet med at tre ikke er et isotopisk material. Dette i kombinasjon med relativt gammelt treverk og mye skade på de bærende elementene i konstruksjonen gjør det spesielt vanskelig. Det er derfor anbefalt at maks last ikke overskrider 200 kg. Rammen har en maks lastlengde på 3000 mm, en maks lastbredde på 1500 mm og en maks lasthøyde på 1300 mm. 
Rammen er i tillegg utstyr med en pute festet til rammen der bilen kan hvile på under transport. Denne puten er festet med fire bolter og kan enkelt tas bort for å transportere andre konstruksjoner, eller objekter. 
Rammen er sikret og låst i bakkant og i forkant av bilen. Det er helt nødvendig at begge disse låsingene er aktivert under transport. Bilen festes videre til rammen med lastestropper eller lignende, fester til slikt utstyr er montert på innsiden av rammen. For videre designmessige detaljer henvises det her til vedlagte sammenstillings og arbeidstegninger.

 

