\section{Dagensløsning}
\subsection{Beskrivelse av dagens løsning}

Dagens løsning består av en skinnekonstruksjon boltet til gulvet i hengeren, og en ramme på seks hjul som ruller på denne skinnen. Se bilde under

Låsingen er en enkel mekanisme bestående av to låsepinner som treffer to hylser på baksiden av rammen, og to braketter der to skruer blir ført gjennom for så å bli låst med en bolt. Se bilde under

Rammen er også utstyrt med bein, slik at bilen kan stå på rammen, ute av bilen. Dette er en fordel om rammen skal brukes ved utstillinger og promoteringsoppdrag.  Se bilde under
\subsection{Ulemper med dagens løsning}

En av de største ulempene med dagens løsning er at rammen bilen ikke er låst i vertikalretning når bilen skal inn og ut av henger. Dette medfører en del ekstra arbeid for de involverte og er noe som Eco maraton teamet har vert misfornøyde med.

En annen ulempe er at det er vanskelig for en person å ta ned beina mens han samtidig tar ut rammen. Dette er både et praktiskproblem men det er også problematisk rent sikkerhetsmessig. 

Det er også problematisk at puten bilen ligger på under transport er for høy, slik at det blir vanskelig å få bilen inn i hengeren. Problemet ligger i at rammen puten er festet på er for høy og at bilen må presses ned av en person samtidig som bilen blir ført inn i henger av en annen.

Hjulene på rammen er et av de største irritasjonsmomentene for de daglige brukerne. Hjulene er for små og mangler nødvendig opplagring i form av solide kulelager. Rammen blir derfor eksepsjonelt vanskelig å få ut og inn av hengeren. 

\subsection{Fordeler med dagens løsning}

Den store fordelen med dagens design er at bilen kan bli fraktet i henger både med og uten hjul. Dette medfører at hjulene på bilen ikke er nær bakken under transport. Slik unngås unødvendige belastninger på hjulopphenget, som i utgangspunktet ikke tåler for mye slag, eller plutselige kraftpåkjenninger. 

En annen fordel med dagens design er at det for det meste er laget i tre, dette gjør modifikasjoner relativt enkle da alt er skrudd sammen med treskruer.

\subsection{Pris}
\subsubsection{Under 500 Nkr}
Totalt utdelt budsjett er på kr 5000. Dette er et absolutt krav
\subsection{Funksjonalitet}
\subsubsection{Opplagret for ikke å bikke}
Dagens rammekonstruksjon vil som nevnt tidligere bikke når bilens og rammens tyngdepunkt når en gitt posisjon, lenge før hele bilen er synlig ute av henger. Denne posisjonen vil Eco Maraton teamet at skal være etter at bilen har komt synlig ut av henger. Det skal være mulig å bruke hengeren med rammen til utstilling av bilen. 
\subsubsection{Lett ut og inn}
Rammen skal kunne ved hjelp av en person eller maskinkraft, bli dyttet eller dratt inn og ut av hengeren. Om det velges maskinkraft skal der være et reservesystem som gjør det samme som det maskinelle. 
\subsubsection{Enkel låsing til henger}
På grunn av liten plass i hengeren når bilen er inne er det et krav at rammen skal låsens på en enkel måte, helst uten at personer trenger gå inn i hengeren bak bilen for å låse rammen. 
\subsubsection{Enkel låsing til ramme}
Bilen skal i all hovedsak låses til rammen på utsiden av hengeren. Det blir da ikke så store tekniske krav til denne mekanismen. Et krav er uansett at dette skal kunne utføres på en enkel måte.
\subsubsection{Frakte bil uten hjul}
Rammen skal kunne bli brukt til å frakte bilen med og uten hjulene på. Dette betyr i praksis at hjulene ikke skal være i kontakt med bakken når den er festet på rammen. 
\subsubsection{Bli brukt til utstilling av bilen}
Henger med ramme skal kunne bli brukt til utstilling av bilen, i forskjellige situasjoner. Rammen må derfor være dimensjonert for både å tåle belastningene, men og på en slik måte at henger ikke bikker når bilen står på utsiden på rammen.
\section{Sikkerhet}
\subsubsection{Sikker og stabil låsing til henger}
\subsubsection{Den konstruerte låsingen på rammen til henger skal kunne stå imot harde nedbremsinger og ujevnt underlag uten at bilen tar skade av dette.  }
\subsubsection{Sikker og stabil låsing av bil}
Bilen skal kunne låses til rammen på en slik måte at den også klarer å stå imot de samme påkjennelsene som nevnt ovenfor.
\subsubsection{To personer skal kunne ta bil inn og ut av henger}
Bilen skal kunne tas ut og inn av hengeren uten hjelp av mer enn to personer. Tunge ustabile løft er mindre sikre en lette stabile løft.  
\section{Utseende}
\subsubsection{Stå i stil til bilen}
Siden rammen og hengeren skal bli brukt på utstillinger, stander osv. er det ønske at rammen har et så estetisk riktig utrykk som mulig. Helst bør rammekonstruksjonen stå i stil til resten av hengeren, og ikke minst bilen. 
\section{Vekt}
\subsubsection{Totalvekt inkludert bil under 300 kg}
På grunn av maksimal total nyttelast på hengeren på 600 kg, er det et krav at den totale vekten av både henger og bil ikke overskrider 300 kg. Dette på grunn av at det er ønskelig å frakte tilleggsutstyr i hengeren sammen med bilen.  
\section{Kompatibilitet}
\subsubsection{Kompatibel med Revolve}
Det er et ønske fra Eco maraton teamet å designe rammen på en slik måte at den enten er direkte kompatibel med bilen som Revolve teamet skal lage, eller er kompatibel etter små tilpasninger. 
