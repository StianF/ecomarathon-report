\section{Konsepter}
Etter at det hadde blitt fastlagt hva gruppa skulle gjøre, måtte vi gjøre en
brainstorm rundt hvordan resultatet skulle se ut. Vi fulgte stepladder-metoden,
litt ulike ideer dukket opp.

Erlend:
Noe ala. eksisterende??

Hans:
Tegnet opp et ganske komplett skjermbilde med tid, hastighet, posisjon på banen
og trendkurver for måledata. Kjøredata og tekniske data var tydelig separert.
Bruk av farger for å indikere betydning av verdier var også spesifisert.

Stian:
%??

Tore Egil:
Fokuserte på tid. Tegnet opp et system for å vise tider, splittider, og disse
tidene sammenlignet med idealtiden.

Torgeir:
Hadde fokusert på fart. Konseptet var et speedometer som viste "riktig fart", 
den hastigheten man bør holde for å komme i mål til riktig tid, og hva man
holder relativt til denne. I tillegg var måledata gjemt ut til sidene, og
ville dukke opp kun når det var data av betydning der.


\section{Vurdering}
For å vurdere disse konseptene valgte vi å ta et møte med ecomarathon teamet,
siden vi manglet grunnlag for å sette opp en vurderingsmatrise. Vi snakket
først med Jardar, og tegningen til Hans falt i god smak. Det ble klart at tid
var langt mer viktig enn fart, så hovedskjermbildet ble det avgjort skulle først
og fremst inneholde rundetider og et kart over banen. Hans sin idè om å trende
tekniske data ble kombinert med Torgeir sin ide om å holde disse skjult. 
Tore Egil sin idè om å vise splittider ble etter hvert avvist av Uwe, som mente
disse ville bli vanskelige å implementere, siden man ikke har nøyaktige
avstander på banen.

%Owe kom etter hvert inn og deltok med sine sterke, men diffuse meninger :P
% "Then maybe..."
