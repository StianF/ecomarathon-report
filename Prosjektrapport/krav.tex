\subsection{Krav fra kunde}
\begin{table}[H]
\begin{tabular}{|l|l|c|c|c|}
\hline
 & \textbf{Funksjon} & \textbf{Må} & \textbf{Bør} & \textbf{Kan} \\ \hline
\textbf{Pris} & Under 500 Nkr. & x & & \\
\hline
\multirow{6}{*}{\textbf{Funksjonalitet}} 
 & Opplagret for ikke å bikke & x & & \\
 & Lett ut og inn & x & & \\
 & Enkel låsing til henger & x & & \\
 & Enkel låsing til bil & & x & \\
 & Frakte bil uten hjul & x & & \\
 & Bli brukt til utstilling av bilen & & x & \\
\hline
\multirow{3}{*}{\textbf{Sikkerhet}}
 & Sikker og stabil låsing til henger & x & & \\
 & Sikker og stabil låsing av bil & x & & \\
 & Trengst kun to personer & & x & \\
\hline
\textbf{Utseende} & Stå i stil til bilen & &  & x \\ \hline
\textbf{Vekt} & Totalvekt inkludert bil under 300 kg & x & & \\ \hline
\textbf{Kompabilitet} & Kompatibel med Revolve & & & x \\ \hline
\end{tabular}
\caption{Funksjonskrav}
\end{table}


\subsubsection{Pris}
\begin{itemize}
\item Under 5000 Nkr \\
Totalt utdelt budsjett er på kr 5000. Dette er et absolutt krav
\end{itemize}
\subsubsection{Funksjonalitet}
\begin{itemize}
\item Opplagret for ikke å bikke \\
Dagens rammekonstruksjon vil som nevnt tidligere bikke når bilens og rammens tyngdepunkt når en gitt posisjon, lenge før hele bilen er synlig ute av henger. Denne posisjonen vil Eco Maraton teamet at skal være etter at bilen har komt synlig ut av henger. Det skal være mulig å bruke hengeren med rammen til utstilling av bilen. 
\item Lett ut og inn \\
Rammen skal kunne ved hjelp av en person eller maskinkraft, bli dyttet eller dratt inn og ut av hengeren. Om det velges maskinkraft skal der være et reservesystem som gjør det samme som det maskinelle. 
\item Enkel låsing til henger \\
På grunn av liten plass i hengeren når bilen er inne er det et krav at rammen skal låsens på en enkel måte, helst uten at personer trenger gå inn i hengeren bak bilen for å låse rammen. 
\item Enkel låsing til ramme \\
Bilen skal i all hovedsak låses til rammen på utsiden av hengeren. Det blir da ikke så store tekniske krav til denne mekanismen. Et krav er uansett at dette skal kunne utføres på en enkel måte.
\item Frakte bil uten hjul \\
Rammen skal kunne bli brukt til å frakte bilen med og uten hjulene på. Dette betyr i praksis at hjulene ikke skal være i kontakt med bakken når den er festet på rammen. 
\item Bli brukt til utstilling av bilen \\
Henger med ramme skal kunne bli brukt til utstilling av bilen, i forskjellige situasjoner. Rammen må derfor være dimensjonert for både å tåle belastningene, men og på en slik måte at henger ikke bikker når bilen står på utsiden på rammen.
\end{itemize}
\subsubsection{Sikkerhet}
\begin{itemize}
\item Sikker og stabil låsing til henger \\
\item Den konstruerte låsingen på rammen til henger skal kunne stå imot harde nedbremsinger og ujevnt underlag uten at bilen tar skade av dette.  
\item Sikker og stabil låsing av bil \\
Bilen skal kunne låses til rammen på en slik måte at den også klarer å stå imot de samme påkjennelsene som nevnt ovenfor.
\item To personer skal kunne ta bil inn og ut av henger \\
Bilen skal kunne tas ut og inn av hengeren uten hjelp av mer enn to personer. Tunge ustabile løft er mindre sikre en lette stabile løft.  
\end{itemize}
\subsubsection{Utseende}
\begin{itemize}
\item Stå i stil til bilen \\
Siden rammen og hengeren skal bli brukt på utstillinger, stander osv. er det ønske at rammen har et så estetisk riktig utrykk som mulig. Helst bør rammekonstruksjonen stå i stil til resten av hengeren, og ikke minst bilen. 
\end{itemize}
\subsubsection{Vekt}
\begin{itemize}
\item Totalvekt inkludert bil under 300 kg \\
På grunn av maksimal total nyttelast på hengeren på 600 kg, er det et krav at den totale vekten av både henger og bil ikke overskrider 300 kg. Dette på grunn av at det er ønskelig å frakte tilleggsutstyr i hengeren sammen med bilen.  
\end{itemize}
\subsubsection{Kompatibilitet}
\begin{itemize}
\item Kompatibel med Revolve \\
Det er et ønske fra Eco maraton teamet å designe rammen på en slik måte at den enten er direkte kompatibel med bilen som Revolve teamet skal lage, eller er kompatibel etter små tilpasninger. 
\end{itemize}
